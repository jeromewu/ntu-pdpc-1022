\documentclass[12pt]{beamer}
\usepackage{graphicx}
\usepackage{color}
\usetheme{Amsterdam}

\title{OpenCL  Homework}
\author{Wen-Chieh Wu}
\date{June 4, 2014}

\begin{document}

\maketitle

\begin{frame}
  \frametitle{Examples}
  There are 3 examples for your references, please copy it by command\\
  \$ cp -r /tmp/opencl.example .
  \begin{itemize}
    {\item hello.world}
    {\item local.max}
    {\item vector.add}
  \end{itemize}
  The programs can be compiled by using command:\\
  \$ make\\
  The programs can be executed by using command:\\
  \$ make run
\end{frame}

\begin{frame}
  \frametitle{Examples}
  \begin{itemize}
  {\item How to compile a opencl program?}\\
  \$ gcc main.c -o main -lOpenCL
  {\item How to execute the program?}\\
  \$ ./main
  \end{itemize}
\end{frame}

\begin{frame}
  \begin{center}
    {\Large Final Homework}
  \end{center}
\end{frame}

\begin{frame}
  \frametitle{Goal}
  \begin{center}
  Implementing Median Filter in OpenCL.
  \end{center}
\end{frame}

\begin{frame}
  \frametitle{What is Median Filter?}
  In signal processing, it is often desirable to be able to perform some kind of noise reduction on an image or signal. The median filter is a nonlinear digital filtering technique, often used to remove noise. 
  \begin{center}
    \includegraphics{img/medianfilter.png}
  \end{center}
\end{frame}

\begin{frame}
  \frametitle{How does it work?}
  Find the median value in the 3x3 window
  \begin{table}
  \begin{tabular}{|c|c|c|} \hline
    1 & 2 & 3 \\ \hline
    4 & 5 & 6 \\ \hline
    7 & 8 & 9 \\ \hline
  \end{tabular}
  \begin{tabular}{|c|c|c|} \hline
    4 & 4 & 5 \\ \hline
    5 & 5 & 6 \\ \hline
    7 & 7 & 8 \\ \hline
  \end{tabular}
  \end{table}
  Ex: 1 2 {\bf 4} 5\\
  Ex: 1 2 3 {\bf 4} 5 6\\
  Ex: 1 2 3 4 {\bf 5} 6 7 8 9
\end{frame}

\begin{frame}
  \frametitle{Input}
  A NxN matrix from binary file, 5 inputs total, and 2 will give you in advance with answer.
  \begin{itemize}
    {\item input.10.dat(1024x1024): 0.465 s}
    {\item input.11.dat(2048x2048): 1.851 s}
    {\item input.12.dat(4096x4096): 7.348 s}
    {\item input.13.dat(8192x8192): 29.32 s}
    {\item input.14.dat(16384x16348): 122.09 s}
  \end{itemize}
  Program execution format:\\
  \$ ./[program name] [inputfile] [N]\\
  EX:\\
  \$ ./medianfilter input.10.dat 10
\end{frame}

\begin{frame}
  \frametitle{Output}
  NxN matrix after median filtering to stdout in plaintext, the matrix is in single line.\\
  Input:\\
  1 2 3 4 5 6 7 8 9\\
  Output:\\
  4 4 5 5 5 6 7 7 8
\end{frame}

\begin{frame}
  \frametitle{Grading}
  If your answer is right, you will get 80\% of the grade, and you will get another 20\% if the execution time is short enough:
  \begin{itemize}
    {\item input.10.dat: 2.37 s}
    {\item input.11.dat: 3.05 s}
    {\item input.12.dat: 4.41 s}
    {\item input.13.dat: 9.52 s}
    {\item input.14.dat: 32.49 s}
  \end{itemize}
\end{frame}

\begin{frame}
  \frametitle{Upload}
  Please upload your program to CEIBA, in the zip file contain a folder of your id and the program main.c, a readme.txt telling me how to compile and other files.\\
  EX:\\
  \begin{itemize}
    {\item r01922003}
          \begin{itemize}
            {\item main.c}
            {\item readme.txt}
            {\item ...}
          \end{itemize}
  \end{itemize}
\end{frame}

\begin{frame}
  \frametitle{Note}
  \begin{itemize}
    {\item Deadline: 6/25 14:00}
  \end{itemize}
\end{frame}

\end{document}
